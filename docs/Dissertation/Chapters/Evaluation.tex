\chapter{Evaluation}
This chapter covers the steps we took to evaluate and test our product (Product Evaluation) and covers any changes we would incur were we conducting this project all over again from the beginning (Project Evaluation).\\
\\
\section{Product Evaluation}
This section covers the testing we conducted in order to test our completed project. Testing was conducted in several parts: to ensure that our product worked correctly and to ensure that our plugin was correctly implemented in IntelliJ.

\subsection{IntelliJ Testing}
We needed to test our plugin worked correctly when installed into IntelliJ. We also needed to check that it installed correctly into IntelliJ. \\

\subsubsection{Operating Systems}
In order to test that our plugin run correctly within IntelliJ, all team members needed to install IntelliJ on their own systems. This allowed us to ensure that our plugin would work with IntelliJ regardless of which operating system it ran on. While the IntelliJ program runs correctly on it's own since it is a professional piece of software, we were unsure if our own plugin would run consistently with the slight changes between Operating Systems. This level of testing was achieved as four of our team mates had access to various installation varieties of Mac OS, with 3 of our team mates having access to linux installations due to other university work requirements, and all team members had access to some version of windows. As such between our team members, we had access to all varieties of operating systems that are offered as downloads for IntelliJ. This allowed our team to check that our plugin worked with all possible versions of IntelliJ, as well as testing that it would work with previous versions thanks to our teams variety in operating system version. All team mates were encouraged to use the plugin and re-install it upon a new feature being implemented or updated. This allowed our team members to compare testing results across systems as well as comparing results with other team members. This allowed us to ensure that our plugin would not fail on users home systems.\\
\\
\subsubsection{IntelliJ implementation}
Our team needed to test that our plugin could be installed into IntelliJ once the features had been added. This allowed us to not only test the new features but also test


It also allowed us to test features similar to a user since the majority of our team members would not have worked on the newly added/updated feature. Our plugin implementation method meant that each member normally worked on an individual feature, moving onto another upon completion. This meant that while our team members knew the specifics of the new feature, they did not know the exact implementation details allowing us to test new features with some degree of programming anomyinity. 
\section{Project Evaluation}
This section covers the difficulties and successes our team had in working together on this project. It evaluates each difficulty and discusses the steps we would incur in order to avoid these difficulties in future. This is different from the previous discussion of problems (Chapter 4 Implementation) as that covered problems we encountered whilst conducting the project, whereas the following problems are those that occurred within the team. The following also covers areas where our team worked well and how we could apply those practises to future projects.  \\
\\
\subsection{Team Problems}

\subsubsection{Locate Team Members}
One of our difficulties was trying to locate our team members. This was due to a lack of information regarding those team members other commitments such as work or appointments. It also did not help that those team members who were not currently available, chose not to inform other team members on the communication channels we had specified. This caused countless problems of work not being completed b the specified date, and other team members needing to 'pick up the slack' in order to finish the project. In future projects we would discover possible future commitments that could affect our project and be more forceful with those members who are not using the agreed communication channels.\\
\\
\subsubsection{Missing Team Member}
From this stemmed the problem of a team member consistently not doing work, due to other commitments, to the effect that we needed to get our client involved. Our project plan was created initially with the understanding that all five team members would commit their time equally such that all team members were equally contributing. However, one of our team members could not be contacted or found for several weeks, meaning that the work he was meant to be completing had to be completed by another member. Our client gave us advice for dealing with this problem and spoke with the team member in order to ensure that he would now contribute. For future projects we would use what we learned from our client to deal with the problem before the project suffer too much due to it. Due to our lack of experience with team projects of this size, we were un-educated on how to deal with a problem such as this without causing problems in the team dynamic. Now we have the solution to such a problem we can apply it much quicker and ensure the work can be completed. \\
\\

\subsection{Team Successes}

\subsubsection{Successful collaboration}
Our team members generally managed to work well together and successfully managed to collaborate on any project problems that occurred. Other than those instances mentioned previously, our team members managed to remain amiable with each other even with the stresses of conducting such a large product on top of the standard stress that comes with being a Computer Science Student. We managed this by discussing general problems we were having with other subjects and finding common ground to create conversation, such as favourite games or areas of the degree we particularly dislike. Due to spending supplementary time on bonding with our team members, when problems did occur we were able to resolve them without generating resentment or raising tempers, which are common in projects such as this. As such we would attempt to spend extra time knowing our team mates in future team projects as it allowed us to concentrate on completing the project together, rather than seeing the team mates as deliberately annoying or problematic.\\
\\
\subsubsection{Communication}
Apart from the team members mentioned previously, on the whole our team members managed to communicate successfully. If we had arranged to have a meeting and a team member developed an illness which prevented them from attending, then that team member would either inform other team members of the problem ,but would also log onto one of our communication methods during the time of the meeting so that they could contribute to the meeting even though they were not physically able to attend. Whilst we did have problems with a couple of team member who did not consistently use the communication methods (see previous section), on a whole our team managed to use the various communication methods available to discuss the project. We also arranged to have multiple forms of communication open, in case one method was temporarily down. This was achieved by deciding on communication methods and ensuring that all team members had accounts for those members and were in the group chat for all methods. For future projects we would insist on multiple communication methods, as it allows those members who prefer video/chat communication such as Skype to continue using it, whilst those who prefer more text based methods such as Facebook messenger to communicate through that. Therefore, by allowing our team member to use their preferred methods, they were more likely to communicate any updates or problems to other team members. 
