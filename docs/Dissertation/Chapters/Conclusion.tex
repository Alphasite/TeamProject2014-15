\chapter{Conclusion}
\label{conclusion}

Our team was originally tasked with developing an Interactive Grammar Toolkit by our client, Dr. Simon Gay. This toolkit was to assist students of Computing Science and grammar developers with the creation and editing of Grammars in a friendly environment. It had to be usable by a variety of users, allow editng and creation of grammars, allow the grammar to be exported to external grammars, and potentially display a graphical diagram of the grammar rules.\\
\\
Our team has successfully created this toolkit by implementing an IntelliJ Plugin, that can be installed on any machine running IntelliJ. The toolkit uses IntelliJ's graphical interface, which is thoroughly tested and very popular amongst coders, in order to assist the user as much as possible. It adds Grammar creation in EBNF to the variety of accepted languages already supported by IntelliJ IDE. The plugin allows creation and editing of grammars, as required, whilst informing the users of syntax,logic and potential runtime errors. This makes our plugin friendly and useful to potential grammar devleopers, espeically students of Computing Science who may not be aware of their mistake such as Left-Recursion. It also currently exports the grammar for both ANTLR and Yacc parser formats, which are very popular. Our grammar also provides a graphical translation of the grammar as an external html file with diagram interpretations of each rule. This allows users to compare their grammars in a graphical format that is cross-platform.\\
\\
User evaluation was very positive in response to our plugin, with several users commenting how useful a tool would have been if it had been available when our own class was studying grammars. As such our team would suggest the possiblity of the plugins details being made to future students, such that they can potentially download and use the plugin to assist in their studies if they so choose.
\\
Our team manged to work well together on this project with a few issues that were eventually dealt with. Upon undertaking a future project our team members would work to avoid these issues and apply our experience gained from this project in solving the root of the problem before a similar issue occurs. Our team also has some potential ideas that we or another third year project team, could implement in order to improve the working of our plugin or potential features to add to increase the usability. Our issues and potential future work can be found in the following sections. 


\section{Future Work}
This section discusses possible future additions our team, or another team, could potentially add to the plugin in future years. 

\subsection{Parser Generator}
In the initial meetings with our client, Dr Simon Gay, he specifically stated that our toolkit was not to parse the grammars created; it was instead to export the grammar into formats that could be parsed by popular parsers such as ANTLR and Yacc. He explained that this was due to time constraints and the level of difficulty in implementing such a feature on top of the already specified toolkit. For a future project team studying Computing Science. this could be a possible feature to implement, as it would allow the team to learn more about how grammars are parsed into a useable format. It would also be a great project for our own members as it would allow our members to revisit our project and implement a feature we had all discussed as a possibility. 

\subsection{Code Formatter}
Currently our team members are satisfied with the state of the Code Formatter. However a future project team could possibly work on improving the formatters formatting of the Grammar. This is not a necessity ,however is it a possible additional that a future team may consider in order to improve upon our project implementation. 
